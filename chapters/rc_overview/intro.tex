%!TEX root = ../../thesis.tex

\epigraph{When a person understands a story, he can demonstrate his understanding by answering questions about the story. Since questions can be devised to query any aspect of text comprehension, the ability to answer questions is the strongest possible demonstration of understanding. If a computer is said to understand a story, we must demand of the computer the same demonstration of understanding that we require of people. Until such demands are met, we have no way of evaluating text understanding programs.}{Wendy Lehnert, 1977}


In this chapter, we aim to provide readers with an overview of reading comprehension. We begin with the history of reading comprehension (Section~\ref{sec:rc-history}), from the early systems developed in the 1970s, to the attempts to build machine learning models for this task, to the more recent resurgence of neural (deep learning) approaches. This field has been completely reshaped by neural reading comprehension, and the progress is very exciting.

We then formally define the reading comprehension task as a supervised learning problem in Section~\ref{sec:task-definition} and describe four different categories based on the answer type. We end by discussing their evaluation metrics.

Next we discuss briefly how reading comprehension differs from question answering, especially in their final goals (Section~\ref{sec:rc-qa-diff}). Finally, we discuss that how the interplay of large-scale datasets and neural models contributes to the development of modern reading comprehension in Section~\ref{sec:rc-drive}.

% This chapter is going to cover the following topics:
% \begin{itemize}
% \item
%   Task definition: formalize 4 different types of RC tasks and their evaluation metrics.
% \item
%   Disucss how RC and QA are different
% \item
%   Recap the history of reading comprehension: early systems, machine learning models and the deep learning era.
% \item
%   Finally,
% \end{itemize}
